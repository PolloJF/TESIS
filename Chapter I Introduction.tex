\chapter{background and motivation}
\section{Introduction}

Primarly, this document corresponds to the tesis job about The developement of a complete Through the Earth communication system for underground environments. The need for reliable communication systems in underground environments has become increasingly important in recent years, particularly in the mining and construction industries. Specially in certain situations like emergencies in mines, where communication is critical for the safety of workers and the efficiency of operations. Underground environments present unique challenges for communication systems, including the presence of conductive materials, complex geological structures, and limited line-of-sight propagation. These challenges can lead to significant signal attenuation and interference, making it difficult to establish reliable communication links. Traditional communication methods, such as radio and wired systems, can opperate in normal conditions, but they often fail in extreme environments or material collapse. As a result, there is a growing need for innovative communication solutions that can overcome these challenges and provide reliable connectivity in underground environments.
For this reason, the Through the Earth (TTE) communication system has been proposed as a potential solution for underground communication. Tipically TTE systems utilize low-frequency electromagnetic waves to transmit data through the earth, allowing for communication over long distances and through various geological materials. This technology has the potential to revolutionize underground communication and improve safety in mining and construction industries. However, the development of a complete TTE communication system for underground environments requires a comprehensive understanding of the underlying principles, challenges, and potential applications of the technology.

\section{Motivation}
The motivation relies on the need for reliable communication system in underground environments without previous infrastructure. The possibility of communicate through centenars of meters of rock and solid material, with a simple system that can be implemented in a short time and with limited resources can be a game changer in the mining and other underground activities.  The ability to establish communication links in challenging environments can enhance safety, improve operational efficiency, and facilitate emergency response efforts. Additionally, the development of a complete TTE communication system can contribute to the advancement of communication technologies and provide valuable insights into the challenges and solutions associated with underground environments.
\section{Hipothesis}
Is possible to develop a complete Through the Earth (TTE) communication system for underground environments considering relatively simple hardware and software and limited resources, achieving an useful system to reach text messaging through centenars of meters of rock and solid material.

\section{Objectives}
\subsection{General Objective}
The main objective of this thesis is to develop a complete Through the Earth (TTE) communication system for underground environments, focusing on the design, implementation, and evaluation of the system's components and performance. This includes the development of resonant coils and antennas, signal processing techniques, and communication protocols that are specifically tailored for underground applications. The goal is to create a reliable TTE communication system that can operate effectively in challenging underground conditions, providing a valuable tool for industries such as mining and construction.
\subsection{Specific Objectives}
    \begin{itemize}
    \item Design and implement resonant coils and antennas for TTE communication, optimizing their performance for underground environments.
    \item Develop signal processing techniques to enhance the reliability and efficiency of the system.
    \item Compare different modulation and coding schemes for TTE communication.
    \item Characterization of the underground medium, including the effects of geological materials and environmental factors on signal propagation.
    \item Evaluate the performance of the TTE communication system in various underground conditions, including different geological materials and environmental factors.
    \item Investigate potential applications of TTE communication systems in mining and construction industries, focusing on safety and operational efficiency.
    \end{itemize}
\section{Document Structure}
\begin{itemize}
    \item Chapter 1: Background and Motivation. This chapter provides an overview of the motivation behind the development of a TTE communication system, the challenges associated with underground communication, and the objectives of the thesis.
    \item Chapter 2: Theoretical Framework. This chapter presents the theoretical principles underlying TTE communication systems, including the physical principles of transmission by magnetic induction, resonant coils and antennas, and fundamentals of digital communication systems.
    \item  Chapter 3: Metodology. This chapter outlines the methodology used in the development of the TTE communication system, including the design and implementation of hardware and software components, as well as the experimental setup for performance evaluation.
    \item Chapter 4: Analysis and Results. This chapter presents the whole development of the TTE communication system, including design, construction, implementation of software models, simulation snd experimental results, and performance evaluation in various underground conditions.
    \item Chapter 5: Discussion. This chapter discusses the implications of the results obtained in the previous chapter, including the challenges faced during the development process and the potential applications of TTE communication systems in underground environments.
    \item Chapter 6: Conclusions and Future Work. This chapter summarizes the main findings of the thesis, highlights the contributions made to the field of TTE communication systems, and suggests potential directions for future research and development in this area.
\end{itemize}


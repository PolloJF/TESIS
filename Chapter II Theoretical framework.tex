% ======================= CHAPTER II =======================
\chapter{Theoretical Framework}\label{ch:theoretical-framework}

% NOTE: Original headings have been merged and expanded.
% Original top-level sections mapped as follows:
% - TTE Communication Systems -> Section 2.1
% - Physical Principles of transmission by Magnetic Induction -> Section 2.2
% - Characterization of the Medium -> Section 2.3
% - Resonant coils and antennas for TTE -> Section 2.5
% - Fundamentals of Digital Communication Systems -> Section 2.6
% - DSP in TTE -> Section 2.10

% ------------------------------------------------------------------
\section{Overview of Through-The-Earth (TTE) Magnetic Induction Communication}

En este capítulo se presentan los fundamentos teóricos y el marco conceptual para lograr comunicación efectiva a través de la roca. Se abordan los principios físicos de la inducción magnética, las características del medio geológico, el modelado del canal, el diseño de transductores (bobinas), y los fundamentos de la comunicación digital aplicados a sistemas TTE. Además, se discuten técnicas de procesamiento de señales digitales (DSP) relevantes y se identifican los principales desafíos técnicos asociados con la comunicación TTE mediante inducción magnética.
\label{sec:overview-mI-TTE}
\subsection{Definition and Scope of TTE MI Communication}
% (From: What is a TTE Communication System?)
% TODO: Define TTE, near-field magnetic induction vs radiative propagation.

\subsection{System Architecture}\label{subsec:system-architecture}
% (From: TTE Communication System Architecture)
% TODO: High-level block diagram: Transmitter coil, power amplifier, matching network, channel (rock), receiver coil, LNA, filtering, demodulation, decoding.

\subsection{Motivation and Application Domains}\label{subsec:motivation-applications}
% (From: Motivation and Applications)
% TODO: Safety, mining, drilling, rescue operations, structural monitoring, underground sensing.

\subsection{Key Technical Challenges (Preview)}\label{subsec:key-challenges-preview}
% (From: Main technical challenges, brief intro; detailed in Section \ref{sec:challenges-limitations})
% TODO: Severe attenuation, limited bandwidth, alignment sensitivity, coil size constraints, interference, power budget.

\subsection{State of the Art Overview}\label{subsec:state-of-art}
% (From: State of the art)
% TODO: Summarize major research milestones, typical frequency ranges (kHz--MHz), achieved ranges, data rates.

\subsection{Comparison with Alternative Underground Technologies}\label{subsec:comparison-alternative-tech}
% (From: TTE and other technologies for underground communication)
% TODO: Compare MI vs RF (higher freq), acoustic, ELF, seismic, optical (impractical), leaky feeder, wired.

\subsection{Comparison with Conventional RF in Lossy Media}\label{subsec:overview-mi-vs-rf}
% (High-level comparison; deeper analysis in Section \ref{subsec:detailed-mi-rf-comparison})
% TODO: Penetration advantages of quasi-static magnetic fields, reduced susceptibility to multipath, bandwidth trade-offs.

% ------------------------------------------------------------------
\section{Electromagnetic Fundamentals for MI Links}\label{sec:em-fundamentals}
% (From: Physical Principles of transmission by Magnetic Induction)


Ampère's Law, in the context of magnetostatics, provides the fundamental relationship between a circulating magnetic field (B) and the electric current (J) that produces it. Specifically, the differential form states that the curl of the magnetic field is directly proportional to the current density \cite{griffiths2023introduction}:

\begin{equation}
    \label{equation_of_ampere_law}
    \nabla \times B = \mu_{0} J
\end{equation}

This law is particularly useful for calculating magnetic fields when physical symmetry is present, serving a role analogous to Gauss's Law in electrostatics. However, this formulation holds strictly only for steady currents, as applying the divergence to the magnetostatic version reveals an inconsistency when charge density is changing over time. James Clerk Maxwell resolved this theoretical flaw by incorporating the displacement current term:

\begin{equation}
    \label{equation_of_ampere_maxwell}
    \nabla \times B = \mu_{0} J + \mu_{0} \epsilon_{0} \frac{\partial E}{\partial t}
\end{equation}

This extended form, known as the Ampère-Maxwell Law, accounts for time-varying electric fields and is essential for describing electromagnetic wave propagation. In the context of magnetic induction communication systems, this law underpins the generation and behavior of magnetic fields produced by time-varying currents in transmitting coils.

%Add equation for Magnetic flux at a distance D from a circular coil of radius a, N turns, carrying current I:

According to Faraday's law, the voltage induced by a magnetic field that goes through a conductive closed loop depends on the temporal variation of the magnetic flux that enters the loop orthogonally. As a result, the voltage induced at the antenna due to the magnetic field is given by:

\begin{equation}
    \label{equation_of_magnetic_induction}
    \begin{split}
        V_{rx}(\omega) &= -j\omega N_{rx}\int_{S}\mu H\cdot dS \\
                       &= -j\omega\mu N_{rx}S_{rx}H \cos(\varphi)
    \end{split}
\end{equation}

where $N_{rx}e~S_{rx}$ are the number of turns and the area of the receiving loop, respectively, and is the angle between the magnetic field and the loop axis, that is orthogonal to its plane. 

La aproximación más simple para el campo magnético creado por una pequeña antena de bucle (usando una aproximación de campo cuasi-estático) es:

\begin{equation}
    \label{equation_of_magnetic_field}
    H_{qe}=\frac{m_{d}}{4\pi r^{3}}\{2~cos(\theta)\hat{r}+sin(\theta)\hat{\theta}\}
\end{equation}

where $m_{d}=N_{tx}I_{tx}\pi a^{2}$ is the magnetic dipole moment of the transmitting coil, $r$ is the distance from the coil center to the observation point, and $\theta$ is the angle between the coil axis and the line connecting the coil center to the observation point. 

\cite{carreno2016through}.

\subsection{Magnetic Induction Fundamentals}\label{subsec:magnetic-induction-fundamentals}
% (From: Magnetic Induction)
% TODO: Faraday's law, mutual inductance basics, coupling coefficient.

\subsection{Quasi-Static Magnetic Field Regime}\label{subsec:quasi-static}
% (From: Cuasistatic Magnetic Field)
% TODO: Conditions for quasi-static approximation (r << lambda/2π), near-field decay (1/r^3) components.

\subsection{Magnetic Flux, Mutual Inductance and Coupling Coefficient}\label{subsec:mutual-inductance}

The mutual inductance $(M)$ between two coils (Loop 1 and Loop 2) is defined as the constant of proportionality relating the magnetic flux ($\psi$) passing through the secondary coil $(S)$ to the current $(I)$ flowing in the primary coil (P), such that:

\begin{equation}
    \label{equation_of_mutual_inductance}
    M = \frac{\psi_{S}}{I_{P}}
\end{equation}

$M$ is a purely geometric parameter that depends on the shape, size, of the two coils. For two parallel circular coils centered on a single axis, where $r_P$ and $r_s$ are the primary and secondary radii, $N_P$ and $N_S$ are the number of turns, and $d$ is the distance separating them, M is approximated by:
\begin{equation}
    M = \frac{\mu N_{P} N_{S} \pi r_{P}^{2} r_{S}^{2}}{2 (r_{P}^{2} + d^{2})^{3/2}}
\end{equation}
%%only if rs << rP and rs??
The coupling coefficient $(k)$ is a dimensionless parameter that quantifies the strength of the magnetic coupling between the two coils, defined as.  It is inherently proportional to the efficiency of the link. In loosely coupled systems, the coupling coefficient is typically small, often around 10 \cite{agbinya2022principles}. $k$ It is formally defined in terms of the mutual inductance (M) and the self-inductances ($L_1$ and $L_2$) of the two coils: 

\begin{equation}
    \label{equation_of_coupling_coefficient}
    k = \frac{M}{\sqrt{L_{1} L_{2}}}
\end{equation}




\subsection{Transfer Function of Inductive Coupling model}\label{subsec:transfer-function-inductive-coupling}

\begin{equation}
    H(s) = \frac{V_{out}(s)}{V_{in}(s)} = \frac{s M R_L}{(sL + R + \frac{1}{sC})^2 + (sM)^2 + (sL + R + \frac{1}{sC}) R_L}
\end{equation}

Where $M$ is the mutual inductance between the two coils, $R_L$ is the load resistance at the receiver side, $R$ is the resistance of each coil, $L$ is the inductance of each coil and $C$ is the capacitance of each coil.

% TODO: Derive M for coaxial circular loops, mention misalignment effects (appendix option).

\subsection{Skin Depth and Frequency-Dependent Attenuation in Rock}\label{subsec:skin-depth}
% TODO: δ = \sqrt{2/(\omega \mu \sigma)}; impact on optimal frequency selection.

\subsection{Propagation Models in Conductive Geological Media}\label{subsec:propagation-models}
% (From: Magnetic propagation models in conductive media)
% TODO: Analytical models, homogeneous vs layered media, empirical fits.

\subsection{Equivalent Circuit Representations}\label{subsec:equivalent-circuit}
% (From: Equivalent Circuit Model)
% TODO: Series/parallel RLC, reflected impedance, transformer analogy.

\subsection{Energy Transfer and Power Budget Foundations}\label{subsec:energy-transfer}
% TODO: Link budget formulation in magnetic domain, transmit current, induced voltage, received power, SNR.

\subsection{Detailed Comparison with RF Approaches}\label{subsec:detailed-mi-rf-comparison}
% (From: Comparison of MI with Conventional RF communication in lossy media)
% TODO: Loss mechanisms: conductive attenuation vs dielectric/ multipath; near-field stability vs fading.

% ------------------------------------------------------------------
\section{Geological Medium Characterization}\label{sec:geological-medium}
% (From: Characterization of the Medium)
\subsection{Electrical Properties of Rock: Conductivity, Permittivity, Permeability}\label{subsec:rock-properties}
% TODO: Typical ranges, anisotropy.

\subsection{Heterogeneity, Stratification and Anisotropy}\label{subsec:heterogeneity}
% TODO: Impact on channel variability; probabilistic modeling.

\subsection{Temperature, Moisture and Mineral Content Effects}\label{subsec:environmental-effects}
% TODO: Parameter sensitivity.

\subsection{Effective Medium and Mixing Models}\label{subsec:mixing-models}
% TODO: Maxwell-Garnett, Bruggeman approximations.

\subsection{Measurement Techniques for Material Parameters}\label{subsec:material-measurement}
% TODO: Lab sample measurements, impedance spectroscopy, inductive probes.

% ------------------------------------------------------------------
\section{Magnetic Induction Channel Modeling}\label{sec:channel-modeling}
\subsection{Geometric Configuration: Coil Orientation, Separation, Misalignment}\label{subsec:geometry}
% TODO: Rotation matrices, coupling dependence.

\subsection{Lumped-Element Channel Models}\label{subsec:lumped-models}
% TODO: Mutual inductance + loss resistance representation.

\subsection{Extended / Semi-Analytical or Distributed Models}\label{subsec:extended-models}
% TODO: When lumped fails; frequency-dependent media parameters.

\subsection{Channel Frequency Response H(f) and Impulse Response}\label{subsec:channel-response}
% TODO: Derive |H(f)| for resonant matched coils.

\subsection{Noise Sources: Thermal, Geological, External EMI}\label{subsec:noise-sources}
% TODO: Noise modeling assumptions (AWGN + colored components).

\subsection{SNR, Link Margin and Capacity Considerations}\label{subsec:snr-capacity}
% TODO: Shannon capacity adaptation to narrowband MI.

% ------------------------------------------------------------------
\section{Transducer (Coil) and Resonant Antenna Design}\label{sec:coil-design}
% (From: Resonant coils and antennas for TTE)
\subsection{Coil Geometry: Radius, Turns, Wire Gauge}\label{subsec:coil-geometry}
% TODO: Trade-offs: inductance vs resistance.

\subsection{Core Materials: Air-Core vs Ferrite}\label{subsec:core-materials}
% TODO: Permeability impact, saturation concerns.

\subsection{Resonance, Quality Factor (Q) and Bandwidth}\label{subsec:q-bandwidth}
% TODO: Q-limited symbol rate, detuning sensitivity.

\subsection{Impedance Matching and Drive Circuits}\label{subsec:matching-networks}
% TODO: L-match, transformer coupling, class D/E/F amplifiers.

\subsection{Magnetic Field Pattern and Spatial Decay}\label{subsec:field-pattern}
% TODO: 1/r^3 near-field decay, transition region.

\subsection{Power Amplification, Drive Current and Efficiency}\label{subsec:power-amplification}
% TODO: Current limits, thermal constraints.

% ------------------------------------------------------------------
\section{Digital Communication Fundamentals for MI Systems}\label{sec:digital-fundamentals}
% (From: Fundamentals of Digital Communication Systems)
\subsection{Signal Representation and Baseband vs Passband Considerations}\label{subsec:signal-representation}
% TODO: Narrowband approximation.

\subsection{Sampling, Aliasing and Bandlimitation Constraints}\label{subsec:sampling}
% TODO: Coil bandwidth vs sampling rate.

\subsection{Noise Modeling and Detection Basics}\label{subsec:noise-detection-basics}
% TODO: Energy detection foundations.

% ------------------------------------------------------------------
\section{Non-Coherent FSK Modulation for MI Communication}\label{sec:noncoherent-fsk}
\subsection{Rationale for Selecting Non-Coherent FSK}\label{subsec:rationale-fsk}
% TODO: Robustness to phase instability; simple receiver.

\subsection{Signal Space and Frequency Separation}\label{subsec:frequency-separation}
% TODO: Δf selection vs symbol duration.

\subsection{Demodulation Techniques: Filterbanks, Energy Detectors}\label{subsec:demodulation}
% TODO: Implementation complexity trade-offs.

\subsection{BER Performance in AWGN}\label{subsec:ber-awgn}
% TODO: Closed-form BER expressions (appendix reference).

\subsection{Impact of Channel Frequency Selectivity and Detuning}\label{subsec:selectivity-detuning}
% TODO: Off-resonance degradation.

% ------------------------------------------------------------------
\section{Error Control Coding: Reed-solomon}\label{sec:error-control-coding}

Reed-Solomon (R-S) codes constitute a powerful class of nonbinary cyclic codes. These codes utilize symbols composed of m-bit sequences, where m is any positive integer greater than two. A conventional R-S code is designated as R-S $(n,k)$, where k represents the number of data symbols being encoded, and n is the total number of code symbols in the resulting block. The relationship defining a conventional R-S code is $(n, k) = (2^{m}-1, 2^{m}-1-2t)$, where $2t = n-k$ is the number of parity symbols, and t is the code's symbol-error correcting capability \cite{sklar2001reed}.

\subsection{Role of FEC in MI Links}

TTE communication systems, often necessary for underground mines, face channels characterized by large attenuation and operate using very low frequencies (VLF). This difficult scenario imposes severe restrictions on the communication system design, including dealing with high levels of atmospheric and electrical equipment noise,. This environment leads to extremely limited bandwidth and capacity in the TTE channels,. Due to the presence of non-Gaussian noise and the low capacity, the use of higher-order digital modulations is often challenging \cite{ma2025through}.

In this context, Forward Error Correction (FEC) techniques, such as Reed-Solomon coding, become essential. FEC allows the system to detect and correct errors that occur during transmission without the need for retransmission, which is particularly beneficial in TTE scenarios where latency and reliability are critical. By adding redundancy to the transmitted data, FEC improves the robustness of the communication link against noise and interference, thereby enhancing the overall performance and reliability of TTE MI communication systems.




\label{subsec:role-fec}
% TODO: Reliability vs power.

\subsection{Block Codes (Hamming, BCH, Reed--Solomon)}\label{subsec:block-codes}
% TODO: Use cases (short frame protection).

\subsection{Convolutional Codes and Viterbi Decoding}\label{subsec:convolutional}
% TODO: Latency vs performance.

\subsection{Modern Codes (LDPC, Polar) Suitability}\label{subsec:modern-codes}
% TODO: Complexity vs embedded constraints.

\subsection{Interleaving for Burst Error Mitigation}\label{subsec:interleaving}
% TODO: Channel memory scenarios.

\subsection{Coding Gain and Trade-offs}\label{subsec:coding-gain}
% TODO: Example link budget improvement.

% ------------------------------------------------------------------
\section{System-Level Synchronization and Framing}\label{sec:synchronization}
\subsection{Preamble Design and Frequency Stabilization}\label{subsec:preamble}
% TODO: Non-coherent requirements.

\subsection{Symbol Timing Recovery}\label{subsec:timing-recovery}
% TODO: Energy envelope methods.

\subsection{Packet Structure and Overhead}\label{subsec:packet-structure}
% TODO: Header, payload, FEC, CRC.

% ------------------------------------------------------------------
\section{Measurement and Channel Characterization Techniques}\label{sec:measurement-techniques}
\subsection{Laboratory Experimental Setups}\label{subsec:labb-setup}
% TODO: Rock analogues, coil fixtures.

\subsection{Transfer Function Measurement (Swept-Sine, MLS, Chirp)}\label{subsec:transfer-measurement}
% TODO: Pros/cons of excitation signals.

\subsection{Field Strength Measurement (Magnetometers, Search Coils)}\label{subsec:field-strength}
% TODO: Calibration methods.

\subsection{Calibration and Error Sources}\label{subsec:calibration}
% TODO: Alignment, temperature drift, instrumentation noise.

\subsection{Parameter Extraction (Mutual Inductance, Effective Conductivity)}\label{subsec:parameter-extraction}
% TODO: Curve fitting strategies.

\subsection{Uncertainty Analysis and Repeatability}\label{subsec:uncertainty}
% TODO: Statistical treatment.

% ------------------------------------------------------------------
\section{Performance Metrics and Evaluation Framework}\label{sec:performance-metrics}
\subsection{BER, FER, Throughput and Latency}\label{subsec:basic-metrics}
% TODO: Definitions and measurement.

\subsection{Energy per Bit and Efficiency Metrics}\label{subsec:energy-metrics}
% TODO: E_b/N_0 vs P_tx constraints.

\subsection{Range--Data Rate--Reliability Trade Space}\label{subsec:trade-space}
% TODO: Pareto analysis concept.

\subsection{Robustness to Misalignment and Orientation}\label{subsec:robustness-misalignment}
% TODO: Orientation statistics.

% ------------------------------------------------------------------
\section{Challenges and Limitations}\label{sec:challenges-limitations}
% (Expands on subsection Key Technical Challenges)
\subsection{Severe Attenuation and Limited Bandwidth}\label{subsec:attenuation-bandwidth}
\subsection{Coil Size vs Portability Constraints}\label{subsec:coil-size}
\subsection{Frequency Drift and Detuning}\label{subsec:frequency-drift}
\subsection{Coupling Variability and Misalignment}\label{subsec:coupling-variability}
\subsection{Interference and Regulatory Considerations}\label{subsec:interference}
\subsection{Detection System}


% ------------------------------------------------------------------
\section{Signal Processing (DSP) Techniques in TTE MI Systems}\label{sec:dsp-tte}
% (From: DSP in TTE)
\subsection{Filtering and Noise Reduction}\label{subsec:filtering-noise}
% TODO: Narrowband filters, adaptive filtering.

\subsection{Adaptive Thresholding and Detection}\label{subsec:adaptive-detection}
% TODO: Non-coherent detection improvements.

\subsection{Channel Estimation / Tracking Approaches}\label{subsec:channel-estimation}
% TODO: Estimating effective gain via pilots.

\subsection{Time/Frequency Synchronization Enhancements}\label{subsec:dsp-synchronization}
% TODO: Sliding correlation.

\begin{equation}
    \label{eq:Correlation}
    z_i(T) = \int_{0}^{T} r(t)s_i(t) \, dt, \quad i=1, \dots, M
\end{equation}

The version for digital signals is given by:
\begin{equation}
    \label{eq:Correlation_dig}
    z_i[n] = \sum_{k=0}^{N-1} r[k]s_i[k+n], \quad i=1, \dots, M
\end{equation}

\subsection{Resource-Constrained Implementation Considerations}\label{subsec:resource-constraints}
% TODO: MCU vs FPGA vs ASIC trade-offs.

% ------------------------------------------------------------------
\section{Emerging Enhancements and Research Directions}\label{sec:emerging}
\subsection{Adaptive Frequency and Power Control}\label{subsec:adaptive-frequency}
\subsection{Multi-Coil Arrays and MIMO Concepts}\label{subsec:multi-coil}
\subsection{Hybrid MI--Acoustic or MI--RF Schemes}\label{subsec:hybrid-schemes}
\subsection{Machine Learning for Channel Estimation and Detection}\label{subsec:ml-channel}
\subsection{Ultra-Low-Power Hardware Advances}\label{subsec:ultra-low-power}

% ------------------------------------------------------------------
\section{Summary of Theoretical Foundations}\label{sec:summary-theory}
% TODO: Recap key points, lead into next chapter.

% ------------------------------------------------------------------
% Optional Appendices for this chapter (if not placed at end of thesis)
% \appendix
% \section{Derivation of Mutual Inductance for Misaligned Circular Loops}
% \section{Approximate BER Expressions for Non-Coherent M-ary FSK}
% \section{Skin Depth Tables for Representative Rock Types}
% \section{Measurement Instrumentation Calibration Notes}
% ------------------------------------------------------------------

% ======================= CHAPTER IV =======================
\chapter{Results}\label{ch:results}

\section{Hardware Developement}

El hardware del sistema de comunicación TTE fue diseñado y construido para cumplir con un conjunto específico de requisitos y restricciones.
En primer lugar el sistema debe ser capaz de satisfacer la prueba de concepto logrando comunicación a una diatancia media pero significativa en un entorno subterráneo.
Asismismo el equipo no debe ser extremadamente voluminoso o pesado para facilitar su transporte y versatilidad en su uso.
\subsection{Design}

El sistema de comunicación TTE está compuesto por varios elementos que trabajan en conjunto para lograr la transmisión y recepción de datos mediante inducción magnética.
El elemento principal es la bobina resonante, la cual está diseñada para generar y recibir campos magnéticos en una frecuencia específica. Además de la bobina, 
el sistema incluye un amplificador para aumentar la potencia del mensaje transmitido y un sistema de switching para intercambiar la conexión de la bobina resonante entre los canales de entrada y salida del sistema.
\subsection{Coil}
La bobina resonante corresponde a al elemento encargado de transmitir el campo magnético variable al circular corriente por su estrcutura (Transmisor) y de entregar la fem inducida en presencia al atravesar por su estructira las líneas de campo magnético transmitido (Receptor). En vista de lograr el objetivo de comunicación a distancia significativa, la bobina debe ser diseñada para maximizar la eficiencia de la transferencia de energía magnética entre el transmisor y el receptor, es por esto que se optó en primera instancia por un diseño simple toroidal, donde las variables de diseño y construcción se reducen a número de vueltas $N$, diámetro de la estructura $D$, y el calibre del conductor $C$.
\newp
Para esta primera versión se fijó el diámetro de la bobina en 55 cm, ya que correspondía a un tamaño manejable y portátil. El calibre del conductor se seleccionó en función de su costo, resistencia eléctrica y peso, optando por un cable de cobre esmaltado de 17 AWG. A continuación se muestra una tabla de resistencia eléctrica en función del calibre del conductor, entregada por el fabricante.

\begin{table}[H]
    \centering
    \begin{tabular}{|c|c|c|}
    \hline
    \textbf{Calibre (AWG)} & \textbf{Diámetro (mm)} & \textbf{Resistencia (ohm/m)} \\ \hline
    10                     & 2.588                  & 0.00328                      \\ \hline
    12                     & 2.053                  & 0.00521                      \\ \hline
    14                     & 1.628                  & 0.00829                      \\ \hline
    16                     & 1.291                  & 0.01317                      \\ \hline
    17                     & 1.150                  & 0.01660                      \\ \hline
    18                     & 1.024                  & 0.02100                      \\ \hline
    20                     & 0.812                  & 0.03340                      \\ \hline
    \end{tabular}
    \caption{Resistencia eléctrica en función del calibre del conductor.}
    \label{tab:resistance-table}
\end{table}
%Ajustar valores segun tabla real.

De esta manera al escoger tal calibre, el cual corresponde a 1.15 mm de diámetro y considerando una masa total cercana a los $1.5 Kg$, se logra una extensión total de 156 metros, lo que se traduce en $N=90$, con esta extensión la resistencia total de la bobina resulta en 2.6 ohmios. Sin embargo al ser medida de forma experimental el valor de resistencia total es de $3 \Omega$, un valor ideal considerando la resistencia a la cual opera el amplificador clase D seleccionado para el proyecto.
\newp
Ya definida la geometría y características de la bobina, se procede a calcular los parámetros eléctricos asociados a la misma. En primera instancia se calcula la inductancia de la bobina utilizando la fórmula para una bobina toroidal:
\begin{equation}
    L = \frac{\mu N^2 A}{l}  %%Comprobar
\end{equation}
Donde:
\begin{itemize}
    \item $L$ es la inductancia en Henrios (H).
    \item $\mu$ es la permeabilidad del núcleo (para aire, $\mu_0 = 4\pi \times 10^{-7} H/m$).
    \item $N$ es el número de vueltas.
    \item $A$ es el área de la sección transversal del toroide en metros cuadrados ($m^2$).
    \item $l$ es la longitud media del camino magnético en metros (m).
\end{itemize}
Considerando un área de sección transversal aproximada de $A = 0.2375 m^2$ y una longitud media del camino magnético de $l = 1.7 m$, se obtiene una inductancia teórica de $L \approx  H$. Sin embargo, al medir la inductancia de la bobina construida utilizando un medidor LCR, se obtiene un valor experimental de $L_1 = 10.67  mH$ y $L_2 = 10.55 mH$. Esta discrepancia puede atribuirse a factores como la distribución no uniforme del campo magnético, las pérdidas en el conductor y las tolerancias en la construcción de la bobina.
\newp
Finalmente, se calcula la capacitancia necesaria para sintonizar la bobina a la frecuencia de resonancia deseada, que en este caso es de 3.2 kHz. La frecuencia de resonancia $f_0$ de un circuito LC está dada por la fórmula:
\begin{equation}
    f_0 = \frac{1}{2\pi \sqrt{LC}}
\end{equation}
Despejando para la capacitancia $C$, se obtiene:
\begin{equation}
    C = \frac{1}{(2\pi f_0)^2 L}
\end{equation}

De esta menera la capacitancia necesaria para sintonizar la bobina a $\approx 3.2 kHz$ resulta en $C \approx 237 \mu F$ para $L_1$ y $C \approx 240 \mu F$ para $L_2$. Se seleccionaron capacitores cerámicos de 470 $\mu F$  dispuesto en serie para alcanzar el valor deseado, considerando la tolerancia de los componentes. A continuación se muestra una figura de la bobina construida:

\begin{figure}[H]
    \centering
    \includegraphics[width=0.5\textwidth]{img/coil.png}
    \caption{Bobina resonante construida para el sistema TTE.}
    \label{fig:coil}
\end{figure}

Al medir la función de transferencia para en un sistema donde se tiene una de las bobinas sintonizadas y otra más pequeña sin sintonizar puede observarse en la figura \ref{fig:transfer-function} para $L_1$ y $L_2$ que la frecuencia de resonancia se encuentra cercana a los 3.2 kHz, cumpliendo con el diseño inicial.
\begin{figure}[H]
    \centering
    \includegraphics[width=0.6\textwidth]{img/TF_L1_L2.png}
    \caption{Función de transferencia del sistema de bobinas resonantes.}
    \label{fig:transfer-function}
\end{figure}

Se muestra a continuación la función de transferencia del sistema completo medido a una distancia de 2 metros entre las bobinas.




\subsection{Amplifier}

El amplificador escogido para la aplicación corresponde a un amplificador de audio clase D basado en los chips TPA3116D2 y NE5532 de Texas Instruments. Este amplificador es capaz de entregar una potencia máxima de 150W Mono, considerando una tensión de 26 V.
\newp 
La elección de este amplificador se basa en su bajo costo, disponibilidad en el mercado y su alta eficiencia al corresponder a un amplificador conmutado, lo que minimiza las pérdidas de potencia y la generación de calor. Además como se enunció anteriormente el sistema operará en un rango de frecuencia en torno a los 3.2 kHz, dentro del espectro audible. A continuación se muestra una figura del amplificador utilizado:

\begin{figure}[H]
    \centering
    \includegraphics[width=0.5\textwidth]{img/amp.png}
    \caption{Amplificador de audio clase D TPA3116D2.}
    \label{fig:amplifier}
\end{figure}


\subsection{Switching Circuit}
\subsection{Other Considerations}
\section{Software Developement}
\subsection{Modulation system}
\subsubsection{Non-coherent FSK}
 The detection system works as a typical non-coherent FSK receiver. 
The incoming signal is filtered at the beggining to reduce the noise influence in terms os  operations in the time domain.
%is there any best way of describing this? What are the others benefits of filtering the signas with a digital bandpassfilter,if are any ?
Then, the signal is sampled by an ADC and processed digitally to extract the information. A model of thw whole system is presented in Figure \ref{fig:detection-system}.
\begin{figure}[H]
    \centering
    \includegraphics[width=0.8\textwidth]{img/Diagrama_Receptor.png}
    \caption{Block diagram of the non-coherent FSK detection system.}
    \label{fig:detection-system}
\end{figure}

Why a Non coherent FSK system?
The non-coherent FSK system is selected due to its robustness to phase variations and frequency offsets %(Include bibiliography about this) 
In comparisson for example models based on PSK or QAM, that are more sensitive to phase noise and frequency offsets. FSK modulation allows the receiver to detect the transmitted symbols based on energy detection in specific frequency bands, without requiring precise phase synchronization. This characteristic makes non-coherent FSK particularly suitable for TTE communication systems, where the channel conditions can be highly variable and unpredictable.
Here is a diagram of the non-coherent FSK modulation and demodulation process in Figure \ref{fig:fsk-mod-demod}.

\begin{figure}[H]
    \centering
    \includegraphics[width=0.8\textwidth]{img/Diagrama_Receptor.png}
    \caption{Block diagram of the non-coherent FSK modulation and demodulation process.}
    \label{fig:fsk-mod-demod}
\end{figure}

As I mentiones brewly the last paragraph, at the beggining the digital signal in filtered and then passes to a Window of len N, that is an important number, the dimensions of this Window
determine the cost of the processing process, that's because the bigger the window is, the main process that's comes right after is the correlation.

\subsubsection{Correlation Process}

Correlation is a mathematical operation that measures the similarity between two signals as a function of the time-lag applied to one of them %Include Bibliography. 
In the context of signal processing, correlation is often used to detect the presence of a known signal (template) within a received signal that may be corrupted by noise.
In the non-coherent FSK detection system, the correlation process is used to compare the received signal with predefined reference signals corresponding to each of the FSK frequencies. 
The correlation here is performs between the received signal (or a portion of it) and a known preamble.
This preamble is a specific sequence of symbols that is transmitted at the beginning of each data packet. The purpose of the preamble is to help the receiver synchronize with the incoming signal and accurately detect the start of the data transmission. %Include Reference)
Here we have several main design parameters: an optimar lenght of preamble, its structure, and optimal use of our computation resources, thats because as bigger the lenght of correlation inputs, more calculus to do in each earing time. %Improve redaction
The size of the window is key of the times that the function \textit{Correlation by parts} is called. 
The output of the correlation process is a correlation coefficient that indicates the degree of similarity between the received signal and the reference signal at different time lags. A high correlation coefficient at a specific time lag suggests that the known signal is present in the received signal at that time.
% Include bibliography
The correlation formula (\ref{eq:Correlation}), given in previous chapter, is used as a the main operation of correlation IQ System.
\begin{figure}
    \centering
    \includegraphics[width=0.8\textwidth]{img/Diagrama_Receptor.png}
    \caption{Block diagram of the correlation by parts process.}
    \label{fig:correlation-by-parts}
\end{figure}

In the last diagram is clear that the incoming signal is partitioned in several parts M, where M is the dimension of the preamble in terms of symbols quantity.
Each section named $S_N$ in correlationed with the $"N"$ symbol of the preamble. The structure of $S_N$ is defined as the size of a symbol $M$ plus the sample difference of between preamble (P) and the window size (S). As greater is this difference , the peak of the correlation (the point that the whole structure of the preamble is detected with the system ) can be detected with correlation a bigger instance of earing. That means that we are working with a major amount of data and real-time signal chunks. 
That could work for example if the criteria of detection del preamble are multiple identifications of these peaks levels. In our model we try to minimize the amount of data to process, to reduce time of calculation of time domain chunks, If we exceed the time that ,
\subsubsection{Ortogonality and Bandwidth}
\subsubsection{Preamble design}
\subsubsection{Decimation?}
\subsection{Performance metrics}
\section{Testing and Validation}
\subsection{Radiolink simulation}
\subsection{Induction Pattern}
\subsection{Measurements underground}

\section{Results Summary}

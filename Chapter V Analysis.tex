% ======================= CHAPTER V =======================
\chapter{Analysis}\label{ch:analysis}

\section{Hardware developement}

\section{Software developement}



\section{Measurements analysis}


\section{Mine measurements}

Here the results obtained on section 4.3.4 will be analyzed. The performance of the proposed communication system called MAGIC was proofed in real operational environment. The first three lines of communication were no problematic at all, actually, all these distances were too less compared to the two lines of communication in the National Astronomical Observatory (20-40m vs. 100-200m). However, the last measurement at 120m between deep in mine and outside was particularly challenging.
\newp
First of all, was complicated to find the correct alineation between the two coils. Due to the missinformation about the exact geographical location of the two emplacements, the team had to determine the best relative angle by trial and error. Of course these misalignments affected the performance of the system, as it was explained in section 2.2. Here is a first error source that could be improved in future tests.
\newp